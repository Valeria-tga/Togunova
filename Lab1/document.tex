% !TeX spellcheck = ru_RU
\documentclass[twoside,11pt]{article}
\usepackage{titlesec}
\usepackage{enumitem}
\usepackage[T2A,T1]{fontenc}
\usepackage[utf8]{inputenc}
\usepackage[russian]{babel}
\usepackage{setspace}
\usepackage{amsmath,amssymb,amsthm}
\usepackage{fancyhdr}
\pagestyle{fancy}
\fancyhf{}
\renewcommand{\headrulewidth}{0pt}
\setlength{\headheight}{15pt}

% 1-inch margins
\textheight 8.4in

\parindent 0in
\parskip 1.5ex

%table
\newcommand{\eps}{\varepsilon}
\newcommand{\inprod}[1]{\left\langle #1 \right\rangle}


\newcommand{\handout}[5]{
\noindent
\begin{center}
	\framebox{
		\vbox{
			\hbox to 5.78in { {\bf Научно-исследовательская практика} \hfill #2 }
			\vspace{4mm}
			\hbox to 5.78in { {\Large \hfill #5  \hfill} }
			\vspace{2mm}
			\hbox to 5.78in { {\em #3 \hfill #4} }
		}
	}
\end{center}
\vspace*{4mm}
}

\newcommand{\lecture}[4]{\handout{#1}{#2}{#3}{#4}{Cистема верстки LaTeX  #1}}

\begin{document}
\lecture{}{Лето 2020}{}{Тогунова Валерия}	
\newpage	
\renewcommand{\headrulewidth}{0pt}
\setcounter{page}{323}
\fancyhead[LO,RE]{\textbf{\thepage}} %номер страницы слева сверху слева на нечетных и справа на четных
\fancyhead[CO,CE]{\textbf{Бесконеччные непрерывные дроби}} %текст-центр-нечетные, текст-центр-четные
\fancyhead[LE]{\textbf{Раздел 13-4}} %текст-слева-четные
\fancyhead[RO]{\textbf{Глава 13}} %текст-справа-нечетные
	
\noindent{целое положительное число. Когда \emph{e} заменяется последовательным разложением, то оно становится}\\[-4pt]
\[
N = 
\frac{1}{n+1} +
\frac{1}{(n+1)(n+2)} +
\frac{1}{(n+1)(n+2)(n+3)} + \cdots
\]
\[
< \frac{1}{n+1} +
\frac{1}{(n+1)(n+2)} +
\frac{1}{(n+2)(n+3)} + \cdots 
\]
\[
=\frac{1}{n+1}+
\left(\frac{1}{n+1}-
\frac{1}{n+2}\right)+
\left(\frac{1}{n+2}-
\frac{1}{n+3}\right)+\cdots=\frac{2}{n+1}<1.
\]\\[-8pt]
\noindent{Поскольку неравенство $0<N<1$ невозможно для целых чисел, то \emph{e} должно быть иррациональным. Поиск истинного происхождения числа \emph{$\pi$} создает большие трудности; Дж. Х. Ламберт (1728-1777), в 1761, передал в Берлинскую академию строгое доказательство иррациональности числа \emph{$\pi$}.}\\
\indent{Учитывая, что число \emph{x} иррационально, стоит задаться вопросом, с какой точностью оно может быть выражено рациональным числом. Один из способов решения этой проблемы -- рассмотреть все рациональные числа с фиксированным знаменателем $\emph{b}>0$. Поскольку \emph{x} лежит между двумя рациональными числами, к примеру возьмем, \emph{$c\backslash b<x<(c+1)/b$}, то отсюда следует, что} \\[-13pt]
\[
\bigg| x-\frac{c}{b} \bigg| < \frac{1}{b}.
\]\\[-8pt]
\noindent{Можно перезаписать это так}\\[-8pt]
\[
\bigg| x-\frac{a}{b}\bigg| < \frac{1}{2b},
\]\\[-8pt]
\noindent{где \emph{a=c} или \emph{a=c+1} любой вариант допустим. Данная непрерывная дробь позволила нам доказать результат, который значительно усиливает последнее записанное неравенство, а именно: для любого иррационального числа \emph{x} существует бесконечно много рациональных чисел $\emph{a/b}$ в младших членах, которые удовлетворяют}\\[-8pt]
\[
\bigg| x - \frac{a}{b} \bigg| < \frac{1}{b^2}
\]\\[-8pt]
\noindent{В действительности, согласно следствию из Теоремы 13-11, любая подходящая дробь \emph{$p_{n}/q_{n}$} непрерывного дробного разложения \emph{x} может быть рациональным числом вида \emph{$a/b$}. В следующей теореме утверждается, что подходящая дробь \emph{$p_n/q_n$} может быть наиболее приближенной к значению \emph{x} среди всех рациональных чисел \emph{$a/b$} со знаменателем равным \emph{$q_n$} или меньше.}\\
\indent{Для более ясного понимания основного смысла данной теоремы рассмотрим следующую лемму.}\\\\
\noindent{\textsc{Лемма.} \emph{Пусть $p_n/q_n$ будет n-я подходящая дробь цепной дроби, представляющей иррациональное число x. Если a и b являются целыми числами, причем $1\le b<q_{n+1}$, то}}\\[-8pt]
\[
\bigg|q_nx-p_n\bigg| \le \bigg|bx-a\bigg|. 
\]\\[-8pt]
\noindent{\textit{Доказательство:} Рассмотрим систему уравнений} \\[-11pt]
\[p_n\alpha+p_{n+1}\beta=a,\]
\[q_n\alpha+q_{n+1}\beta=b.\]\\[-14pt]
\noindent{Определитель, состоящий из коэффициентов данной системы уравнений $p_nq_{n+1}-q_np_{n+1} = (-1)^{n+1}$, имеет единственное решение }\\[-8pt]
\[\alpha=(-1)^{n+1}(aq_{n+1}-bp_{n+1}),\]
\[\beta=(-1)^{n+1}(bp_n-aq_n).\]\\[-14pt]
\noindent{Стоит заметить, что $\alpha \ne0$. В действительности при $\alpha=0$ получаем следующее равенство $aq_{n+1}=bp_{n+1}$ и, поскольку НОД$(p_{n+1},q_{n+1})=1$, то $q_{n+1}|b$ или $b\ge q_{n+1}$, что противоречит нашей гипотезе. В случае $\beta=0$, неравенство, сформулированное в лемме, очевидно выполняется. Из того, что $\beta=0$, следуют следующие равенства $a=p_n\alpha$, $b=q_n\alpha$ и в итоге,}\\[-8pt]
\[
|bx-a|=|\alpha||q_nx-p_n|\ge |q_nx-p_n|.
\]\\[-15pt]
\noindent{Таким образом, в дальнейшем мы можем предполагать, что $\beta\ne 0$.}\\
\indent{Когда $\beta\ne 0$, утверждаем, что $\alpha$ и $\beta$ имеют противоположные знаки.
Если $\beta<0$, то уравнение $q_n\alpha=b-q_{n+1}\beta$ показывает, что $q_n\alpha>0$, следовательно, $\alpha>0$. С другой стороны, если $\beta>0$, то $b<q_{n+1}$, а это означает, что $b<\beta q_{n+1}$, поэтому $\alpha q_n=b-q_{n+1}\beta<0$; из этого вытекает $\alpha<0$. Мы также утверждаем, что, поскольку \emph{х} стоит между подходящими дробями $p_n/q_n$ и $p_{n+1}/q_{n+1}$,}\\[-18pt]
\begin{spacing}{0.2}
	\begin{center}
		$q_nx-p_n$ \qquad и \qquad $q_{n+1}x-p_{n+1}$
	\end{center}
\end{spacing}
\noindent{будут иметь противоположные знаки. Смысл этого рассуждения заключается в том, что числа}\\[-18pt]
\begin{spacing}{0.2}
	\begin{center}
		$\alpha(q_n)x-p_n$ \qquad и \qquad $\beta(q_{n+1}x-p_{n+1})$
	\end{center}
\end{spacing}
\noindent{будут иметь одинаковые знаки; следовательно, модуль их суммы равен сумме их модулей. Именно этот немало важный факт позволяет нам быстро завершить доказательство:}\\[-8pt]
\[
|bx-a|=
|(q_n\alpha+q_{n+1}\beta)x-(p_n\alpha+p_{n+1}\beta)|\]
\[=|\alpha(q_nx-p_n)+\beta(q_{n+1}x-p_{n+1})|\]
\[=|\alpha||q_nx-p_n|+|\beta||q_{n+1}x-p_{n+1}|\]
\[>|\alpha||q_nx-p_n|
\ge |q_nx-p_n|,
\]\\[-13pt]
\noindent{это и есть требуемое неравенство.}\\
\indent{Подходящие дроби $p_n/q_n$ являются лучшим способом для вычисления приблизительного значения иррационального числа \emph{x}, то есть любое другое рациональное число с таким же или с наименьшим знаменателем отличается от \emph{x} на максимальную величину}\\

\noindent{\textsc{Теорема 13-12.} \emph{Если $1\le b\le q_n$, то рациональное число $a/b$ удовлетворяет}}\\[-5pt]
\[
\bigg|x-\frac{p_n}{q_n}\bigg| \le \bigg|x-\frac{a}{b}\bigg|.
\]\\[-5pt]
\noindent{\textit{Доказательство:} если бы выполнялось}\\[-5pt]
\[
\bigg|x-\frac{p_n}{q_n}\bigg|>\bigg|x-\frac{a}{b}\bigg|,
\]\\[-10pt]
\noindent{то}\\[-10pt]
\[
|q_nx-p_n|=q_n\bigg|x-\frac{p_n}{q_n}\bigg|>b\bigg|x-\frac{a}{b}\bigg|=|bx-a|,
\]
\noindent{что противоречит рассужденю леммы.}\\[6pt]
\indent{Историки математики уделяли значительное внимание к экспериментам ученных древних цивилизаций, которые пытались  приблизиться к значению числа $\pi$, возможно, потому, что более точные результаты, по-видимому, позволяют судить о математических навыках. Первая зафиксированная попытка оценить число $\pi$ появилась при \mbox{\emph{Измерении Круга}} великим метематиком древней Сиракузии, Архимедом (287-212 до н. э.). По существу, его метод определения значения числа $\pi$ заключался в том, чтобы вписать и описать правильные многоугольники вокруг круга, определить их периметры и использовать данные фигуры в качестве нижних и верхних границ на окружности. Таким образом, используя многоугольник из 96 сторон, он получил два приближения в неравенстве $223/71 <\pi<22/7$.}

\indent{Теорема 13-12 дает представление о том, почему 22/7, также называемое ''Архимедовское значение числа $\pi$'', использовалось часто вместо числа $\pi$; не было дроби с наименьшим знаменателем, которая давала бы лучшее приближение к данному числу. В то время как}
\begin{center}
	$\bigg|\pi-\frac{22}{7}\bigg| \approx 0.0012645$ \qquad и \qquad $\bigg|\pi-\frac{223}{71}\bigg| \approx 0.0007476,$
\end{center}

\noindent{значение Архимеда 223/71, которое не является подходящей дробью $\pi$, имеет знаменатель, превышающий \emph{$q_2=7$.}
Наша теорема говорит нам, что 333/106 (соотношение числа $\pi$, использованного в Европе в 16 веке) будет приближаться к $\pi$ точнее любого рационального числа со знаменателем, меньшим или равным 106; действительно,
\[
\bigg|\pi-\frac{333}{106}\bigg|\approx 0.0000832.
\]}
\noindent{Из-за величины $q_4$=33102, подходящая дробь $p_3/q_3$ = 355/113 позволяет приблизить к числу $\pi$ с максимальной точностью; из следствия к теореме 13-11 имеем
\[
\bigg|\pi-\frac{355}{113}\bigg|<\frac{1}{113\cdot33102}<\frac{3}{10^7}.
\]}
\noindent{Заслуживающее внимания соотношение 355/113 было известно китайскому математику Цу Чунг-чину (430-501); по некоторым причинам, не указанным в его работах, он обозначил 22/7 как 'неточное значение'' числа $\pi$ и 355/113 как ''точное значение''. Последнее соотношение не было общепринятым в Европе до конца 16-го века, пока Адриан Антонисзун не открыл его.} \\
\indent{Это идеальный момент для записи теоремы, которая гласит, что любое "точное" (в подходящем смысле) рациональное приближение к \emph{x} должно быть подходящей дробью для \emph{x}. Была бы некоторая определенность, если бы}
\[
\bigg|x-\frac{a}{b}\bigg|<\frac{1}{b^2}
\]
	
\noindent{где подразумевается, что $a/b=p_n/q_n$ для некоторого \emph{n}; на данный момент излишне, чтобы надеяться на более точное неравенство, которое будет гарантировать тот же вывод.}\\

\noindent{\textsc{Теорема 13-13.} \emph{Пусть x иррациональное число. Если рациональное число a/b, где $b\ge1$ и} НОД\emph{(a,b)=1, удовлетворяет}}
\[
\bigg|x-\frac{a}{b}\bigg|<\frac{1}{2b^2}
\]
		
\noindent{то a/b является одной из подходящих дробей $p_n/q_n$ в представлении непрерывной дроби x.}\\

\noindent{\textit{Доказательство:} Предположим, что \emph{a/b} не сходится к \emph{x}. Зная, что $q_k$ образуют возрастающую последовательность, мы видим, что существует единственное целое число \emph{n}, для которого $q_n\le b<q_{n+1}$. Для этого \emph{n} последняя лемма дает первое неравенство в цепочке
\[
|q_nx-p_n|\le|bx-a|=b\bigg|x-\frac{a}{b}\bigg|<\frac{1}{2b},\]}		
\end{document}