% !TeX spellcheck = ru_RU
\documentclass[twoside,11pt]{article}
\usepackage{titlesec}
\usepackage{enumitem}
\usepackage[T2A,T1]{fontenc}
\usepackage[utf8]{inputenc}
\usepackage[russian]{babel}
\usepackage{setspace}
\usepackage{amsmath,amssymb,amsthm}
\usepackage{fancyhdr}

\pagestyle{fancy}
\fancyhf{}
\renewcommand{\headrulewidth}{0pt}
\setlength{\headheight}{15pt}
\textheight 8.6in
\parindent 0.2in
\parskip 1.5ex

\newtheorem{theorem}{Теорема}
\newtheorem{lemma}{Лемма}

\newcommand{\eps}{\varepsilon}
\newcommand{\inprod}[1]{\left\langle #1 \right\rangle}
\newcommand{\handout}[5]{
	\noindent
	\begin{center}
		\framebox{
			\vbox{
				\hbox to 5.78in { {\bf Научно-исследовательская практика} \hfill #2 }
				\vspace{4mm}
				\hbox to 5.78in { {\Large \hfill #5  \hfill} }
				\vspace{2mm}
				\hbox to 5.78in { {\em #3 \hfill #4} }
			}
		}
	\end{center}
	\vspace*{4mm}
}
\newcommand{\lecture}[4]{\handout{#1}{#2}{#3}{Scribe: #4}{Cистема верстки LaTeX #1}}

\begin{document}
\lecture{}{Лето 2020}{}{Тогунова Валерия}

\newpage	
\renewcommand{\headrulewidth}{0pt}
\setcounter{page}{323}
\fancyhead[LO,RE]{\textbf{\thepage}} %номер страницы слева сверху слева на нечетных и справа на четных
\fancyhead[CO,CE]{\textbf{Бесконеччные непрерывные дроби}} %текст-центр-нечетные, текст-центр-четные
\fancyhead[LE]{\textbf{Раздел 13-4}} %текст-слева-четные
\fancyhead[RO]{\textbf{Глава 13}} %текст-справа-нечетные

\indent{целое положительное число. Когда $e$ заменяется последовательным разложением, то оно становится}
\begin{equation*}
\begin{split}
N = 
\frac{1}{n+1} +
\frac{1}{(n+1)(n+2)} +
\frac{1}{(n+1)(n+2)(n+3)} + \cdots \\
< \frac{1}{n+1} +
\frac{1}{(n+1)(n+2)} +
\frac{1}{(n+2)(n+3)} + \cdots \\
=\frac{1}{n+1}+
\left(\frac{1}{n+1}-
\frac{1}{n+2}\right)+
\left(\frac{1}{n+2}-
\frac{1}{n+3}\right)+\cdots=\frac{2}{n+1}<1.
\end{split}
\end{equation*}
\indent{Поскольку неравенство $0<N<1$ невозможно для целых чисел, то $e$ должно быть иррациональным. Поиск истинного происхождения числа $\pi$ создает большие трудности; Дж. Х. Ламберт (1728-1777), в 1761, передал в Берлинскую академию строгое доказательство иррациональности числа $\pi$.}

\indent{Учитывая, что число $x$ иррационально, стоит задаться вопросом, с какой точностью оно может быть выражено рациональным числом. Один из способов решения этой проблемы -- рассмотреть все рациональные числа с фиксированным знаменателем $b>0$. Поскольку $x$ лежит между двумя рациональными числами, к примеру возьмем, $c\backslash b<x<(c+1)/b$, то отсюда следует, что}
\[
\bigg| x-\frac{c}{b} \bigg| < \frac{1}{b}.
\]
\indent{Можно перезаписать это так}
\[
\bigg| x-\frac{a}{b}\bigg| < \frac{1}{2b},
\]
\indent{где $a=c$ или $a=c+1$ любой вариант допустим. Данная непрерывная дробь позволила нам доказать результат, который значительно усиливает последнее записанное неравенство, а именно: для любого иррационального числа $x$ существует бесконечно много рациональных чисел $a/b$ в младших членах, которые удовлетворяют}
\[
\bigg| x - \frac{a}{b} \bigg| < \frac{1}{b^2}
\]
\indent{В действительности, согласно следствию из Теоремы 13-11, любая подходящая дробь $p_{n}/q_{n}$ непрерывного дробного разложения $x$ может быть рациональным числом вида $a/b$. В следующей теореме утверждается, что подходящая дробь $p_n/q_n$ может быть наиболее приближенной к значению $x$ среди всех рациональных чисел $a/b$ со знаменателем равным $q_n$ или меньше.}

\indent{Для более ясного понимания основного смысла данной теоремы рассмотрим следующую лемму.}

\begin{lemma}
Пусть $p_n/q_n$ будет n-я подходящая дробь цепной дроби, представляющей иррациональное число $x$. Если a и b являются целыми числами, причем $1\le b<q_{n+1}$, то
\end{lemma}
\[
\bigg|q_nx-p_n\bigg| \le \bigg|bx-a\bigg|.
\]
\begin{proof} 
\indent{Рассмотрим систему уравнений}
\begin{align*}
p_n\alpha+p_{n+1}\beta &=a,\\
q_n\alpha+q_{n+1}\beta &=b.
\end{align*}
\indent{Определитель, состоящий из коэффициентов данной системы уравнений $p_nq_{n+1}-q_np_{n+1} = (-1)^{n+1}$, имеет единственное решение }
\begin{align*}
\alpha &=(-1)^{n+1}(aq_{n+1}-bp_{n+1}),\\
\beta &=(-1)^{n+1}(bp_n-aq_n).
\end{align*}
\indent{Стоит заметить, что $\alpha \ne0$. В действительности при $\alpha=0$ получаем следующее равенство $aq_{n+1}=bp_{n+1}$ и, поскольку $\text{НОД}(p_{n+1},q_{n+1})=1$, то $q_{n+1}|b$ или $b\ge q_{n+1}$, что противоречит нашей гипотезе. В случае $\beta=0$, неравенство, сформулированное в лемме, очевидно выполняется. Из того, что $\beta=0$, следуют следующие равенства $a=p_n\alpha$, $b=q_n\alpha$ и в итоге,}
\[
|bx-a|=|\alpha||q_nx-p_n|\ge |q_nx-p_n|.
\]
\indent{Таким образом, в дальнейшем мы можем предполагать, что $\beta\ne 0$.}

\indent{Когда $\beta\ne 0$, утверждаем, что $\alpha$ и $\beta$ имеют противоположные знаки.
Если $\beta<0$, то уравнение $q_n\alpha=b-q_{n+1}\beta$ показывает, что $q_n\alpha>0$, следовательно, $\alpha>0$. С другой стороны, если $\beta>0$, то $b<q_{n+1}$, а это означает, что $b<\beta q_{n+1}$, поэтому $\alpha q_n=b-q_{n+1}\beta<0$; из этого вытекает $\alpha<0$. Мы также утверждаем, что, поскольку $х$ стоит между подходящими дробями $p_n/q_n$ и $p_{n+1}/q_{n+1}$,}
\begin{spacing}{0.2}
	\begin{center}
		$q_nx-p_n$ \qquad и \qquad $q_{n+1}x-p_{n+1}$
	\end{center}
\end{spacing}
\indent{будут иметь противоположные знаки. Смысл этого рассуждения заключается в том, что числа}
\begin{spacing}{0.2}
	\begin{center}
		$\alpha(q_n)x-p_n$ \qquad и \qquad $\beta(q_{n+1}x-p_{n+1})$
	\end{center}
\end{spacing}
\indent{будут иметь одинаковые знаки; следовательно, модуль их суммы равен сумме их модулей. Именно этот немало важный факт позволяет нам быстро завершить доказательство:}
\begin{equation*}
\begin{split}
|bx-a|&=|(q_n\alpha+q_{n+1}\beta)x-(p_n\alpha+p_{n+1}\beta)| \\
&=|\alpha||q_nx-p_n|+|\beta||q_{n+1}x-p_{n+1}|\\
&=|\alpha||q_nx-p_n|+|\beta||q_{n+1}x-p_{n+1}|\\
&>|\alpha||q_nx-p_n|\ge |q_nx-p_n|,
\end{split}
\end{equation*}
\indent{это и есть требуемое неравенство.}
\end{proof}

\indent{Подходящие дроби $p_n/q_n$ являются лучшим способом для вычисления приблизительного значения иррационального числа $x$, то есть любое другое рациональное число с таким же или с наименьшим знаменателем отличается от $x$ на максимальную величину}

\begin{theorem} \label{t1}
\indent{Если $1\le b\le q_n$, то рациональное число $a/b$ удовлетворяет}
\[
\bigg|x-\frac{p_n}{q_n}\bigg| \le \bigg|x-\frac{a}{b}\bigg|.
\]
\end{theorem}
\begin{proof} 
\indent{Если бы выполнялось}
\[
\bigg|x-\frac{p_n}{q_n}\bigg|>\bigg|x-\frac{a}{b}\bigg|,
\]
\indent{то}
\[
|q_nx-p_n|=q_n\bigg|x-\frac{p_n}{q_n}\bigg|>b\bigg|x-\frac{a}{b}\bigg|=|bx-a|,
\]
\indent{что противоречит рассужденю леммы.}
\end{proof}

\indent{Историки математики уделяли значительное внимание к экспериментам ученных древних цивилизаций, которые пытались  приблизиться к значению числа $\pi$, возможно, потому, что более точные результаты, по-видимому, позволяют судить о математических навыках. Первая зафиксированная попытка оценить число $\pi$ появилась при \mbox{\emph{Измерении Круга}} великим метематиком древней Сиракузии, Архимедом (287-212 до н. э.). По существу, его метод определения значения числа $\pi$ заключался в том, чтобы вписать и описать правильные многоугольники вокруг круга, определить их периметры и использовать данные фигуры в качестве нижних и верхних границ на окружности. Таким образом, используя многоугольник из $96$ сторон, он получил два приближения в неравенстве $223/71 <\pi<22/7$.}

\indent{ Теорема \ref{t1} дает представление о том, почему $22/7$, также называемое ''Архимедовское значение числа $\pi$'', использовалось часто вместо числа $\pi$; не было дроби с наименьшим знаменателем, которая давала бы лучшее приближение к данному числу. В то время как}
\begin{center}
	$\bigg|\pi-\frac{22}{7}\bigg| \approx 0.0012645$ \qquad и \qquad $\bigg|\pi-\frac{223}{71}\bigg| \approx 0.0007476,$
\end{center}
\indent{значение Архимеда $223/71$, которое не является подходящей дробью $\pi$, имеет знаменатель, превышающий $q_2=7$. Наша теорема говорит нам, что $333/106$ (соотношение числа $\pi$, использованного в Европе в 16 веке) будет приближаться к $\pi$ точнее любого рационального числа со знаменателем, меньшим или равным $106$; действительно,}
\[
\bigg|\pi-\frac{333}{106}\bigg|\approx 0.0000832.
\]
\indent{Из-за величины $q_4=33102$, подходящая дробь $p_3/q_3 = 355/113$ позволяет приблизить к числу $\pi$ с максимальной точностью; из следствия к теореме 13-11 имеем
\[
\bigg|\pi-\frac{355}{113}\bigg|<\frac{1}{113\cdot33102}<\frac{3}{10^7}.
\]}
\indent{Заслуживающее внимания соотношение $355/113$ было известно китайскому математику Цу Чунг-чину (430-501); по некоторым причинам, не указанным в его работах, он обозначил $22/7$ как 'неточное значение'' числа $\pi$ и $355/113$ как ''точное значение''. Последнее соотношение не было общепринятым в Европе до конца 16-го века, пока Адриан Антонисзун не открыл его.}

\indent{Это идеальный момент для записи теоремы, которая гласит, что любое "точное" (в подходящем смысле) рациональное приближение к $x$ должно быть подходящей дробью для $x$. Была бы некоторая определенность, если бы}
\[
\bigg|x-\frac{a}{b}\bigg|<\frac{1}{b^2}
\]
\indent{где подразумевается, что $a/b=p_n/q_n$ для некоторого $n$; на данный момент излишне, чтобы надеяться на более точное неравенство, которое будет гарантировать тот же вывод.}

\begin{theorem} 
	\indent{Пусть $x$ иррациональное число. Если рациональное число $a/b$, где $b\ge1$ и $\text{НОД}(a,b)=1$, удовлетворяет}
\[
\bigg|x-\frac{a}{b}\bigg|<\frac{1}{2b^2}
\]
\indent{то $a/b$ является одной из подходящих дробей $p_n/q_n$ в представлении непрерывной дроби $x$.}
\end{theorem}
\begin{proof}
	\indent{Предположим, что $a/b$ не сходится к $x$. Зная, что $q_k$ образуют возрастающую последовательность, мы видим, что существует единственное целое число $n$, для которого $q_n\le b<q_{n+1}$. Для этого $n$ последняя лемма дает первое неравенство в цепочке}
\[
|q_nx-p_n|\le|bx-a|=b\bigg|x-\frac{a}{b}\bigg|<\frac{1}{2b},
\]
\end{proof}
\end{document}